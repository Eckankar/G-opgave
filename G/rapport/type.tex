\chapter{Type checker}

The type checker is implemented in \texttt{Type.sml}. The following changes
were made made to implement the remainder of the Cat language:

\begin{enumerate}
    \item {
        The remaining types were added to \texttt{checkType}.
    }
    \item {
        The remaining patterns were added to \texttt{checkPat}, which
        checks a pattern and returns a symbol table for variables.
        For \texttt{Cat.TupleP} (\autoref{fig:type:tuplep}), it was
        necessary doing a recursive check using \texttt{foldr}. This is
        handled by first retrieving the list of types specifying the
        tuple and then comparing it recursively to the input.
    }
    \item {
        The majority of the changes were made in \texttt{checkExp}, which does
        type checking on a Cat expression. The remaining expressions were
        added to ensure that expressions are valid according to the semantics
        in Cat.
        \begin{enumerate}
        \item {
            We've added handling of boolean expressions. This covers the
            constants which just resolve to a bool. We've also added operators
            returning bools such as \texttt{and}, \texttt{or}, etc. These
            have been structured in the same way as the arithmetic operations:
            First we evaluate the two expressions and then we compare them
            resulting in either a bool or an error.
        }
        \item {
            We've also added case expressions. These are made by using the 
            \texttt{checkMatch} auxilliary function. This used our additions to
            \texttt{checkPat} to check the patterns correctly.
        }
        \item {
            \texttt{let} expressions have also been implemented by using the
            semantic definition. We simply convert them to a case expression
            using an auxilliary function, \texttt{letToCase}, which has been
            included as \autoref{fig:type:lettocase}.
        }
        \item {
            We check the type of \texttt{Cat.MkType} by looking up the list of
            types in the symbol table and compare it to the argument.
        }
        \end{enumerate}
    }
    \item {
        A symbol table containing the type declarations is made using the
        \texttt{getTyDecs} function. These are then checked for validity by
        \texttt{checkTyDec}. Calls to these functions have been added to
        \texttt{checkProgram}, which is also declared in \texttt{Type.sig} and
        is the main function for the type checker.
    }
\end{enumerate}
\begin{figure}
    \centering
    \begin{lstlisting}
    | (Cat.TupleP (ps, pos), TyVar t) =>
        let
          val ts = (case lookup t ttable of
                      SOME types => types
                    | NONE => raise Error ("Non-existing type "^t, pos))
        in
          if List.length ts = List.length ps
          then
            ListPair.foldr (fn (p, t, vs) =>
              combineTables (checkPat p t ttable pos) vs pos
            ) [] (ps, ts)
          else raise Error ("Tuple size doesn't match type", pos)
        end
    \end{lstlisting}
    \caption{Type checking of tuple patterns.}
    \label{fig:type:tuplep}
\end{figure}
\begin{figure}
    \centering
    \begin{lstlisting}
  (* Rewrites a let statement to a nested sequence of case statements *)
  fun letToCase [] e0 pos = e0
    | letToCase ((p, e, pos)::decs) e0 pos =
      Cat.Case (e, [(p, letToCase decs e0 pos)], pos)
    \end{lstlisting}
    \caption{\texttt{letToCase}: converts a \texttt{let}-expression to a nested series of \texttt{case}-expressions.}
    \label{fig:type:lettocase}
\end{figure}
