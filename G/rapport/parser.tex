\chapter{Parser}
The parser is autogenerated with \mono{mosmlyac} from the file 
\mono{Parser.grm}.

In order to make the parser work with the whole of Cat we have done some
additions:

\begin{description}
\item [Added tokens] We have added a lot of tokens (ie. \mono{TRUE},
        \mono{FALSE}, \mono{CASE}, \mono{LET}, etc.).
\item [Added associativity] With the addition of comparisons, we've also had
        to assign associativity to these operators.
\item [Added types] We've added a type for \mono{Cat.TyDec} lists and
        \mono{Cat.Dec}.
\item [Added productions] A lot of productions had to be added to support all
        of Cat. We added these productions precisely as they were written in
        the grammar. This did however give a couple of shift/reduce conflicts.
        These are listed below:
        \begin{itemize}
        \item \textbf{Ambiguous lists -} A lot of the productions were on the
                form $Types \to Types , Types$ or $Pats \to Pats , Pats$.
                These caused shift/reduce conflicts, but could be resolved
                easily changing them to the form $Types \to Type , Types$.
                (This doesn't keep the associativity though, but the end result
                is the same.)\\
                This was also the case in $Dec$ and $Exps$.
        \item \textbf{Tuples in $EXP$ -} When declaring a tuple with
                $Exp \to ( Exps ) : id$ it gave a shift/reduce conflict because
                of the other production $Exp\to ( Exp )$. The parser didn't
                know if it should reduce the ``list'' inside the the braces or
                if it should shift the next parenthesis. We solved this by
                splitting $Exp \to ( Exps ) : id$ in two: One with the list,
                and one with a single element.
        \end{itemize}
\end{description}
