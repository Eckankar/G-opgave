\chapter{Tests}
In order to thoroughly test our type checker we found that a few testcases
were missing. These are shown below:

\begin{figure}
    \centering
    \begin{lstlisting}
        type body = (head, torso, legs)

        x = @ : body
    \end{lstlisting}
    \label{fig:tests:error01}
    \caption{Error01: Declaration of a type composed of nonexisting types.}
\end{figure}

\begin{figure}
    \centering
    \begin{lstlisting}
        fun add : int -> int
          x => x+y
        end

        add 5
    \end{lstlisting}
    \label{fig:tests:error02}
    \caption{Error02: Function body uses an undeclared variable.}
\end{figure}


\section{Debugging of compilers}
In the processing of testing our compiler we ran into problems with the qsort
and treesort examples. In order to debug these more easily we partitioned the
treesort code into several smaller tests.

\begin{figure}
    \centering
    \begin{lstlisting}
        type list = (int, list)

        fun print : list -> int
          @      => 0
        | (e, l) =>  let a = write e in print l
        end

        print (1, (2, (3, @:list):list):list):list
    \end{lstlisting}
    \label{fig:tests:test01}
    \caption{Test01: Simple printing of a list. A more isolated testcase than
             the predefined ones.}
\end{figure}

\begin{figure}
    \centering
    \begin{lstlisting}
        type tree = (tree, int, tree)

        fun printTree : tree -> int
            (l, v, r) => let a = printTree l;
                             b = write v;
                             c = printTree r in 0
          | @ => 0
        end

        printTree ((@:tree, 1, @:tree):tree, 2, (@:tree, 3, @:tree):tree):tree
    \end{lstlisting}
    \label{fig:tests:test02}
    \caption{Test02: Simple printing of a tree. A more isolated testcase than
             the predefined ones.}
\end{figure}

\begin{figure}
    \centering
    \begin{lstlisting}
        type tree = (tree, int, tree)
        type list = (int, list)
        type tl = (tree, list)

        fun flatten : tl -> list
            (@,l) => l
          | ((t1,n,t2),l) => flatten(t1,(n,flatten(t2,l):tl):list):tl
        end

        fun print : list -> int
            (v, l) => let a = write v in print l
          | @ => 0
        end

        let
          ls = flatten (((@:tree, 1, @:tree):tree, 2, (@:tree, 3, @:tree):tree):tree, @:list):tl
        in
          print ls
    \end{lstlisting}
    \label{fig:tests:test03}
    \caption{Test03: Flattens a tree and prints it.}
\end{figure}

\begin{figure}
    \centering
    \begin{lstlisting}
        type tree = (tree,int,tree)
        type ti = (tree,int)

        fun insert : ti -> tree
            (@,n) => (@:tree, n, @:tree):tree
          | ((t1,m,t2),n) =>
               if n<m then (insert(t1,n):ti,m,t2):tree
               else (t1,m,insert(t2,n):ti):tree
        end

        insert (((@:tree, 1, @:tree):tree, 2, (@:tree, 6, @:tree):tree):tree, 4):ti
    \end{lstlisting}
    \label{fig:tests:test03}
    \caption{Test04: Test of insert from treesort on a predefined tree.}
\end{figure}

\begin{figure}
    \centering
    \begin{lstlisting}
        write if true then 4 else 5
    \end{lstlisting}
    \label{fig:tests:test03}
    \caption{Test05: Test of return values from \texttt{if}-statement.}
\end{figure}
