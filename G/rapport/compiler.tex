\chapter{Compiler}

\section{Representation}
The Cat language has three kinds of types: integers, booleans and user defined
tuples. We have chosen to represent them the following way:

\begin{description}
    \item[int] {
        Integers use their natural representation as signed 32 bit integers.
    }
    \item[bool] {
        Booleans are represented as the values \texttt{0} and \texttt{1}.
    }
    \item[Tuples] {
        We have chosen to represent our tuples like you would arrays of the
        same length on the heap. That is, given a tuple of length $n$, we
        allocate $4n$ bytes on the heap and save the values sequentially in
        this memory. The value returned is then a pointer to the first element
        of the tuple. To represent the value null, we make the natural choice
        of using a null pointer. If a tuple consists of other tuples, these
        represented as their addresses on the heap.
    }
\end{description}


\section{Specific implementations}
Following are some other design choices we have made:

\begin{description}
\item [New variables] In general we often opt for using new variables, rather
        than reusing old ones. This limits the lifetime of variables, which
        helps limit the amount of registers needed by the register allocator.

\item [Domain of equality] Since the domain of \texttt{=} isn't explicitly
        specified, we chose to make it work only on values of types
        \texttt{int} and \texttt{bool}. This choice was made for the sake of
        simplicity.

\item [Negation] \texttt{not} is implemented in a slightly tricky way, as seen
        in \autoref{fig:compiler:not}. We use the fact that booleans are always
        represented as $0$ or $1$, which means that doing a \texttt{SLTI} with
        $1$ always will flip the number to the opposite one.

\item [Let expressions] Like in the type checker we are doing these by
        converting them to a case expression with an auxilliary function,
        \mono{letToCase}, shown in \autoref{fig:type:lettocase}.

\item [Pattern matching] This is also done much like the type checker. The
        simple types (int, bools, etc.) are evaluated pretty straightforward.
        Tuples are evaluated recursively on all the values of the tuple. We
        also check if the tuple we are comparing with is null before we try to
        read the addresses from the heap.

\item [Shortcircuiting] We've made sure that \texttt{and} and \texttt{or} are
        shortcircuiting. This is done as shown in \autoref{fig:compiler:and}. 
\end{description}

\begin{figure}
    \centering
    \begin{lstlisting}
    | Cat.Not (e, pos) =>
        let
          val t = "_not_"^newName()
          val code = compileExp e vtable t
        in
          code @ [Mips.SLTI (place,t,"1")]
        end
    \end{lstlisting}
    \caption{Implementation of \texttt{not}.}
    \label{fig:compiler:not}
\end{figure}
\begin{figure}
    \centering
    \begin{lstlisting}
    | Cat.And (e1, e2, pos) =>
        let
          val t1 = "_and1_"^newName()
          val t2 = "_and2_"^newName()
          val code1 = compileExp e1 vtable t1
          val code2 = compileExp e2 vtable t2
          val lfalse = "_andf_"^newName()
          val ldone = "_andd_"^newName()
        in
          code1 @ [Mips.BEQ (t1, "0", lfalse)]
                @ code2
                @ [Mips.BEQ (t2, "0", lfalse),
                   Mips.LI (place, makeConst 1),
                   Mips.J ldone,
                   Mips.LABEL lfalse,
                   Mips.LI (place, makeConst 0),
                   Mips.LABEL ldone]
        end
    \end{lstlisting}
    \caption{Implementation of the short-circuiting \texttt{and}.}
    \label{fig:compiler:and}
\end{figure}

% Stuff goes here. Guess we should include code pieces like in the type
% checker. Should probably talk about register allocation as well and our
% translation to MIPS.
